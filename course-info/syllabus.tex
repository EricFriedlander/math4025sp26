% Options for packages loaded elsewhere
\PassOptionsToPackage{unicode}{hyperref}
\PassOptionsToPackage{hyphens}{url}
\PassOptionsToPackage{dvipsnames,svgnames,x11names}{xcolor}
%
\documentclass[
  letterpaper,
  DIV=11,
  numbers=noendperiod]{scrartcl}

\usepackage{amsmath,amssymb}
\usepackage{iftex}
\ifPDFTeX
  \usepackage[T1]{fontenc}
  \usepackage[utf8]{inputenc}
  \usepackage{textcomp} % provide euro and other symbols
\else % if luatex or xetex
  \usepackage{unicode-math}
  \defaultfontfeatures{Scale=MatchLowercase}
  \defaultfontfeatures[\rmfamily]{Ligatures=TeX,Scale=1}
\fi
\usepackage{lmodern}
\ifPDFTeX\else  
    % xetex/luatex font selection
\fi
% Use upquote if available, for straight quotes in verbatim environments
\IfFileExists{upquote.sty}{\usepackage{upquote}}{}
\IfFileExists{microtype.sty}{% use microtype if available
  \usepackage[]{microtype}
  \UseMicrotypeSet[protrusion]{basicmath} % disable protrusion for tt fonts
}{}
\makeatletter
\@ifundefined{KOMAClassName}{% if non-KOMA class
  \IfFileExists{parskip.sty}{%
    \usepackage{parskip}
  }{% else
    \setlength{\parindent}{0pt}
    \setlength{\parskip}{6pt plus 2pt minus 1pt}}
}{% if KOMA class
  \KOMAoptions{parskip=half}}
\makeatother
\usepackage{xcolor}
\setlength{\emergencystretch}{3em} % prevent overfull lines
\setcounter{secnumdepth}{-\maxdimen} % remove section numbering
% Make \paragraph and \subparagraph free-standing
\makeatletter
\ifx\paragraph\undefined\else
  \let\oldparagraph\paragraph
  \renewcommand{\paragraph}{
    \@ifstar
      \xxxParagraphStar
      \xxxParagraphNoStar
  }
  \newcommand{\xxxParagraphStar}[1]{\oldparagraph*{#1}\mbox{}}
  \newcommand{\xxxParagraphNoStar}[1]{\oldparagraph{#1}\mbox{}}
\fi
\ifx\subparagraph\undefined\else
  \let\oldsubparagraph\subparagraph
  \renewcommand{\subparagraph}{
    \@ifstar
      \xxxSubParagraphStar
      \xxxSubParagraphNoStar
  }
  \newcommand{\xxxSubParagraphStar}[1]{\oldsubparagraph*{#1}\mbox{}}
  \newcommand{\xxxSubParagraphNoStar}[1]{\oldsubparagraph{#1}\mbox{}}
\fi
\makeatother


\providecommand{\tightlist}{%
  \setlength{\itemsep}{0pt}\setlength{\parskip}{0pt}}\usepackage{longtable,booktabs,array}
\usepackage{calc} % for calculating minipage widths
% Correct order of tables after \paragraph or \subparagraph
\usepackage{etoolbox}
\makeatletter
\patchcmd\longtable{\par}{\if@noskipsec\mbox{}\fi\par}{}{}
\makeatother
% Allow footnotes in longtable head/foot
\IfFileExists{footnotehyper.sty}{\usepackage{footnotehyper}}{\usepackage{footnote}}
\makesavenoteenv{longtable}
\usepackage{graphicx}
\makeatletter
\newsavebox\pandoc@box
\newcommand*\pandocbounded[1]{% scales image to fit in text height/width
  \sbox\pandoc@box{#1}%
  \Gscale@div\@tempa{\textheight}{\dimexpr\ht\pandoc@box+\dp\pandoc@box\relax}%
  \Gscale@div\@tempb{\linewidth}{\wd\pandoc@box}%
  \ifdim\@tempb\p@<\@tempa\p@\let\@tempa\@tempb\fi% select the smaller of both
  \ifdim\@tempa\p@<\p@\scalebox{\@tempa}{\usebox\pandoc@box}%
  \else\usebox{\pandoc@box}%
  \fi%
}
% Set default figure placement to htbp
\def\fps@figure{htbp}
\makeatother

\KOMAoption{captions}{tableheading}
\makeatletter
\@ifpackageloaded{caption}{}{\usepackage{caption}}
\AtBeginDocument{%
\ifdefined\contentsname
  \renewcommand*\contentsname{Table of contents}
\else
  \newcommand\contentsname{Table of contents}
\fi
\ifdefined\listfigurename
  \renewcommand*\listfigurename{List of Figures}
\else
  \newcommand\listfigurename{List of Figures}
\fi
\ifdefined\listtablename
  \renewcommand*\listtablename{List of Tables}
\else
  \newcommand\listtablename{List of Tables}
\fi
\ifdefined\figurename
  \renewcommand*\figurename{Figure}
\else
  \newcommand\figurename{Figure}
\fi
\ifdefined\tablename
  \renewcommand*\tablename{Table}
\else
  \newcommand\tablename{Table}
\fi
}
\@ifpackageloaded{float}{}{\usepackage{float}}
\floatstyle{ruled}
\@ifundefined{c@chapter}{\newfloat{codelisting}{h}{lop}}{\newfloat{codelisting}{h}{lop}[chapter]}
\floatname{codelisting}{Listing}
\newcommand*\listoflistings{\listof{codelisting}{List of Listings}}
\makeatother
\makeatletter
\makeatother
\makeatletter
\@ifpackageloaded{caption}{}{\usepackage{caption}}
\@ifpackageloaded{subcaption}{}{\usepackage{subcaption}}
\makeatother

\usepackage{bookmark}

\IfFileExists{xurl.sty}{\usepackage{xurl}}{} % add URL line breaks if available
\urlstyle{same} % disable monospaced font for URLs
\hypersetup{
  pdftitle={Syllabus},
  colorlinks=true,
  linkcolor={blue},
  filecolor={Maroon},
  citecolor={Blue},
  urlcolor={Blue},
  pdfcreator={LaTeX via pandoc}}


\title{Syllabus}
\author{}
\date{}

\begin{document}
\maketitle


\subsection{Course info}\label{course-info}

\subsubsection{Class meetings \& Office
Hours}\label{class-meetings-office-hours}

\begin{longtable}[]{@{}
  >{\raggedright\arraybackslash}p{(\linewidth - 4\tabcolsep) * \real{0.3111}}
  >{\raggedright\arraybackslash}p{(\linewidth - 4\tabcolsep) * \real{0.3222}}
  >{\raggedright\arraybackslash}p{(\linewidth - 4\tabcolsep) * \real{0.3667}}@{}}
\toprule\noalign{}
\endhead
\bottomrule\noalign{}
\endlastfoot
\textbf{Lecture} & MWF 3:00 - 3:50pm & Cruzen-Murray Library (CML)
208 \\
\textbf{(Optional) Homework Lab} & W 4:00 - 4:50pm (Tentative) &
Cruzen-Murray Library (CML) 208 \\
\textbf{Office Hours} & M 9:30 - 10:30am & Boone 126B \\
\textbf{Office Hours} & T 9:00 - 10:00am & Boone 126B \\
\textbf{Office Hours} & W 1:30 - 2:30pm & Boone 126B \\
\textbf{Office Hours} & TH 10:00 - 11:00am & Boone 126B \\
\end{longtable}

Office hours are also available by appointment, just email me!

\subsubsection{Instructor Information}\label{instructor-information}

\begin{itemize}
\tightlist
\item
  \textbf{Instructor}: Dr.~Eric Friedlander
\item
  \textbf{Office}: Boone 126B
\item
  \textbf{Email}:
  \href{mailto:efriedlander@collegeofidaho.edu}{\nolinkurl{efriedlander@collegeofidaho.edu}}
\end{itemize}

\subsection{Course Learning
Objectives}\label{course-learning-objectives}

By the end of the semester, you will be able to\ldots{}

\begin{itemize}
\item
  tackle predictive modeling problems arising from real data.
\item
  use Python to fit and evaluate machine learning models.
\item
  assess whether a proposed model is appropriate and describe its
  limitations.
\item
  use Jupyter notebooks and quarto to write reproducible reports and
  GitHub for version control and collaboration.
\item
  effectively communicate results results through writing and oral
  presentations.
\end{itemize}

\subsection{Course community}\label{course-community}

\subsubsection{College of Idaho Honor
Code}\label{college-of-idaho-honor-code}

\begin{quote}
The College of Idaho maintains that academic honesty and integrity are
essential values in the educational process. Operating under an Honor
Code philosophy, the College expects conduct rooted in honesty,
integrity, and understanding, allowing members of a diverse student body
to live together and interact and learn from one another in ways that
protect both personal freedom and community standards. Violations of
academic honesty are addressed primarily by the instructor and may be
referred to the
\href{https://collegeofidaho.smartcatalogiq.com/en/current/Undergraduate-Catalog/Policies-and-Procedures/Academic-Misconduct}{Student
Judicial Board}.
\end{quote}

By participating in this course, you are agreeing that all your work and
conduct will be in accordance with the College of Idaho Honor Code.

\subsubsection{Disability Accommodation
Statement}\label{disability-accommodation-statement}

The College of Idaho seeks to provide an educational environment that is
accessible to the needs of students with disabilities. The College
provides reasonable services to enrolled students who have a documented
permanent or temporary physical, psychological, learning, intellectual,
or sensory disability that qualifies the student for accommodations
under the Americans with Disabilities Act or section 504 of the
Rehabilitation Act of 1973. If you have, or think you may have, a
disability that impacts your performance as a student in this class, you
are encouraged to arrange support services and/or accommodations through
the Department of Accessibility and Learning Excellence located in
McCain 201B and available via email at
\href{mailto:accessibility@collegeofidaho.edu}{\nolinkurl{accessibility@collegeofidaho.edu}}.
Reasonable academic accommodations may be provided to students who
submit appropriate and current documentation of their disability.
Accommodations can be arranged only through this process and are not
retroactively applied. More information can be found on the DALE webpage
(\url{https://www.collegeofidaho.edu/accessibility}).

\subsubsection{Communication}\label{communication}

All lecture notes, assignment instructions, an up-to-date schedule, and
other course materials may be found on the course website,
\href{https://EricFriedlander.github.io/math4025sp26/}{EricFriedlander.github.io/math4025sp26}.

Periodic announcements will be sent via email and will also be available
through Canvas and grades will be stored in the Canvas gradebook. Please
check your email regularly to ensure you have the latest announcements
for the course.

We will be using \textbf{Microsoft Teams} for communication. Please make
sure you have access to the class team.

\subsubsection{In class agreements}\label{in-class-agreements}

If we discuss/agree to something in class or office hours which requires
action from me (e.g.~``you may turn in your homework late due to a
sporting event''), you MUST send me a follow-up message. If you don't, I
will almost certainly forget, and our agreement will be considered null
and void.

\subsubsection{Getting help in the
course}\label{getting-help-in-the-course}

\begin{itemize}
\tightlist
\item
  If you have a question during lecture or lab, feel free to ask it!
  There are likely other students with the same question, so by asking
  you will create a learning opportunity for everyone.
\item
  I am here to help you be successful in the course. You are encouraged
  to attend \emph{office hours} and the \emph{homework lab} to ask
  questions about the course content and assignments. Many questions are
  most effectively answered as you discuss them with others, so office
  hours are a valuable resource. You are encouraged to use them!
\item
  Outside of class and office hours, any general questions about course
  content or assignments can be emailed to me.
\end{itemize}

\subsubsection{Email}\label{email}

If you have questions about assignment extensions or accommodations,
please email
\href{mailto:efriedlander@collegeofidaho.edu}{\nolinkurl{efriedlander@collegeofidaho.edu}}.
Please see \hyperref[late-work-policy]{Late work policy} for more
information. \textbf{If you email me about an error please include a
screenshot of the error and the code causing the error.} Barring
extenuating circumstances, I will respond to course emails within 48
hours Monday - Friday. Response time may be slower for emails sent
Friday evening - Sunday.

Check out the \href{support.qmd}{Support} page for more resources.

\subsection{Textbook}\label{textbook}

The official textbook for this course is:

\begin{itemize}
\item
  An Introduction to Statistical Learning with Applications in Python by
  Gareth James, Daniela Witten, Trevor Hastie, Robert Tibshirani, and
  Jonathan Taylor.

  \begin{itemize}
  \item
    Colloquially referred to as ``ISLP''.
  \item
    It's free!
  \end{itemize}
\end{itemize}

\subsection{Assignments}\label{assignments}

You will be assessed based on six components: homework, quizzes, a job
application, a job interview, a midterm exam, a hack-a-thon, and a
project.

\subsubsection{Quizzes}\label{quizzes}

Videos will be posted on the course website that you are expected to
watch before class. There will be a short quiz at the beginning of class
every Tuesday to ensure you are keeping up with the material.

\subsubsection{Homework}\label{homework}

Homework will be graded on a pass/fail basis and will be assessed via
presentation. Assignments will be due at the beginning of class on
Thursday. Students will be randomly chosen to present their solutions.
Each solution will be followed by a discussion by the class on what did
and didn't go well.

\begin{itemize}
\tightlist
\item
  \textbf{Honesty Policy}: If you did not complete the homework and are
  honest about it, you will receive 50\%. If you demonstrate no
  understanding of the homework during presentation, you will receive a
  0\%.
\end{itemize}

\subsubsection{Job Application \& Job
Interview}\label{job-application-job-interview}

During this course you will apply to one ``job''. I will generate the
job advertisement including a real company and base the job description
on the course content and similar job advertisements. The job
application will have three components:

\begin{enumerate}
\def\labelenumi{\arabic{enumi}.}
\tightlist
\item
  A cover letter.
\item
  A resume.
\item
  A portfolio.
\end{enumerate}

All three of these should be tailored to the job description and the
company to which you are applying. Your portfolio will consist of
self-contained data analyses of your choosing. To create your portfolio,
you will be required to create a website.

After you submit your job application, you will be invited to schedule a
one-hour long job interview. \textbf{It is your job to schedule your job
interview with Dr.~Friedlander.} The interview will include general
interview questions, questions about your portfolio, and a ``case
interview'' portion.

\subsubsection{Midterm Exam}\label{midterm-exam}

There will be one midterm exam to assess your understanding of the core
concepts covered in the first half of the course.

\subsubsection{Hack-a-thon}\label{hack-a-thon}

At some point in the semester we will participate in a ``Hack-a-thon''
as a class. Namely, you will be given a short period of time (1-3 days)
to build a model and make a set of predictions. After the competition is
over, you will be required to present on your model. Part of your score
will be determine by how well your model performs and extra credit will
be given to the top scoring individuals.

\subsubsection{Project}\label{project}

During the latter portion of the course, you will complete a final
project that involves a deep exploration of a problem. More details for
the final project will be provided later in the course.

\subsection{Grading}\label{grading}

The final course grade will be calculated as follows:

\begin{longtable}[]{@{}ll@{}}
\toprule\noalign{}
Category & Percentage \\
\midrule\noalign{}
\endhead
\bottomrule\noalign{}
\endlastfoot
Homework & 10\% \\
Quizzes & 10\% \\
Job Application & 20\% \\
Job Interview & 20\% \\
Midterm Exam & 15\% \\
Hack-a-thon \& Presentation & 10\% \\
Final Project & 15\% \\
\end{longtable}

The final letter grade will be determined based on the following
thresholds:

\begin{longtable}[]{@{}ll@{}}
\toprule\noalign{}
Letter Grade & Final Course Grade \\
\midrule\noalign{}
\endhead
\bottomrule\noalign{}
\endlastfoot
A & \textgreater= 93 \\
A- & 90 - 92.99 \\
B+ & 87 - 89.99 \\
B & 83 - 86.99 \\
B- & 80 - 82.99 \\
C+ & 77 - 79.99 \\
C & 73 - 76.99 \\
C- & 70 - 72.99 \\
D+ & 67 - 69.99 \\
D & 63 - 66.99 \\
D- & 60 - 62.99 \\
F & \textless{} 60 \\
\end{longtable}

\subsection{Course policies}\label{course-policies}

\subsubsection{Academic honesty}\label{academic-honesty}

\textbf{TL;DR: Don't cheat!}

\begin{itemize}
\item
  The job application assignments must be completed individually but you
  are welcome to discuss the assignment with classmates (e.g., discuss
  what's the best way for approaching a problem, what functions are
  useful for accomplishing a particular task, etc.). However you may not
  directly share (i.e.~via copy/paste or copying) any code or prose with
  anyone other than myself.
\item
  For the hack-a-thon, everyone will submit their predictions and give
  their own presentations. However, you are encouraged to work together.
  You are allowed to share code with one another. However, everyone
  should be able to explain what they did and everyone's projects should
  be unique in some way. Point reductions will be given if two
  individuals submit the exact same predictions.
\item
  For the projects, collaboration within teams is not only allowed, but
  expected. Communication between teams at a high level is also allowed
  however you may not share code or components of the project across
  teams.
\item
  \textbf{Reusing code}: Unless explicitly stated otherwise, you may
  make use of online resources (e.g.~StackOverflow) for coding examples
  on assignments. If you directly use code from an outside source (or
  use it as inspiration), you must explicitly cite where you obtained
  the code. Any recycled code that is discovered and is not explicitly
  cited will be treated as plagiarism.
\item
  \textbf{Use of artificial intelligence (AI)}: You are allowed and even
  encouraged to use AI tools (such as ChatGPT, GitHub Copilot, etc.) to
  supplement your learning. However, strict adherence to the following
  is required:

  \begin{itemize}
  \item
    \textbf{You are responsible for all work that you submit.}
  \item
    You must be able to \textbf{explain what every line of code does}
    and \textbf{explain all results} that you submit.
  \item
    If you cannot explain your code or results when asked, you will
    receive a grade of 0 for that assignment.
  \end{itemize}
\end{itemize}

If you are unsure if the use of a particular resource complies with the
academic honesty policy, just ask.

Regardless of course delivery format, it is the responsibility of all
students to understand and follow all College of Idaho policies,
including academic integrity (e.g., completing one's own work, following
proper citation of sources, adhering to guidance around group work
projects, and more). Ignoring these requirements is a violation of the
Honor Code.

\subsubsection{Late work policy}\label{late-work-policy}

The due dates for assignments are there to help you keep up with the
course material and to ensure the teaching team can provide feedback
within a timely manner. I understand that things come up periodically
that could make it difficult to submit an assignment by the deadline.

\begin{itemize}
\item
  \textbf{Late Homework:} Homework is pass/fail based. Assignments are
  due Thursday morning. If you do not complete the assignment but are
  honest about it, you will receive 50\%. If you demonstrate no
  understanding during the presentation, you will receive 0\%.
\item
  \textbf{Quizzes \& Presentations:} Late quizzes and presentations will
  \textbf{not} be accepted without an excused absence.
\item
  \textbf{School-Sponsored Events/Illness:} If an assignment or meeting
  must be missed due to a school-sponsored event, you must let me know
  at least a week ahead of time so that we can schedule a time for you
  to make up the work before you leave. If an assignment or meeting must
  be missed due to illness, you must let me know as soon as it is safe
  for you to do so and before the assignment or meeting if possible.
  Failure to adhere to this policy will result in a 35\% penalty on the
  corresponding assignment.
\end{itemize}




\end{document}
